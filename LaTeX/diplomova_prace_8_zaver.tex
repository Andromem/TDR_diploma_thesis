\chapter{Závěr}

V rámci této práce byla prostudována problematika reflektometrie a různé architektury reflektometrů. Na základě takto získaných poznatků byly teoreticky vybrány použitelné techniky a základní bloky pro vytvoření reflektometru v časové doméně. Z těchto základních prvků byly vybrány jen ty nejvhodnější, případně nejjednodušší na implementaci. Poté již bylo možné navrhnout základní blokovou architekturu reflektometru vzorkujícím v ekvivalentním čase, navrhnout podobu jednotlivých bloků a metody jejich propojení do jednoho celku. 

Základní bloky reflektometru byly odsimulovány a na základě těchto simulací byly zapojení a hodnoty použitých součástek optimalizovány tak, aby co nejlépe plnily svou funkci. Takto optimalizované zapojení bylo překresleno v návrhovém prostředí a převedeno do podoby desky plošných spojů. Po osazení a oživení byl vytvořen základní firmware, kterým bylo již možné zjistit funkčnost celého zařízení. Z dat získaných z této první fáze vývoje byly opraveny simulace a opět byla provedena řada optimalizací. Vzhledem k tomu, že deska plošných spojů byla navržena s předpokladem nezbytnosti této druhé řady optimalizací, bylo možné zapojení snadno upravit. Po této první iteraci oprav se již zapojení ustálilo a nadále byl vyvíjen již jen firmware a software.

Firmware reflektometru byl navržen tak, aby umožnil všechny potřebné autokalibrace při co nejjednodušším ovládání. Je navržen tak, že obsluhu krok po kroku provádí autokalibračními kroky a následně i měřením. Celý průběh autokalibrace i měření je graficky srozumitelně indikován. Pro základní kalibraci úrovní využívá reflektometr pouze kalibr \quotedblbase open\textquotedblleft{} a znalosti vlastního zapojení. Po skončení měření jsou obsluze zobrazeny detekované diskontinuity na vedení. Reflektometr u každé diskontinuity detekuje, o jaký typ diskontinuity se jedná a obsluze sdělí její typ, polohu na vedení a koeficient odrazu odpovídající této diskontinuitě.

Reflektometr je schopen měřit v kroku \SI{20}{\pico\second}, tedy s ekvivalentní vzorkovací frekvencí \SI{50}{\gigasample}. Náběžná hrana budicího signálu je kratší než \SI{90}{\pico\second}, délka náběžné hrany, kterou je reflektometr schopen změřit, činí přibližně \SI{220}{\pico\second}. Analogová šířka pásma reflektometru tak přesahuje \SI{1}{\giga\hertz} a je možné detekovat polohu diskontinuity s rozlišením lepším než \SI{0.3}{\centi\meter}.

Nad rámec zadání byl vytvořen ovládací software pro PC, který umožňuje rozšířené ovládání reflektometru a pohodlnější zobrazení a manipulaci změřených dat.

V rámci práce tedy byl vytvořen funkční reflektometr s jednoduchým ovládáním, který je schopen detekovat základní diskontinuity a jejich typ a polohu. Podařilo se splnit i vlastní cíle, zapojení je vcelku jednoduché a snadno znovuvyrobitelné, nevyžaduje žádné neobvyklé součástky ani speciální pracovní podmínky a je schopno komunikace s počítačem.

Dále by bylo možné reflektometr vylepšit použitím kvalitnějšího konektoru na měřicím portu a opravou vedení vedoucího ke konektoru, případně rozšířením analogové šířky pásma a snížením úrovně šumu.